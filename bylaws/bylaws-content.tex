%%% CONSTITUTION & BYLAWS
%% NAME (n)
\section{Name}
The full name of the Society shall be ``The Mathematics Society of the University
of Waterloo''; its short name shall be ``MathSoc''.

%% OBJECT (o)
\section{Object}
The object of the Society shall be to serve, represent, and promote
undergraduate students in the Faculty of Mathematics at the University of
Waterloo by providing services to students; encouraging and co-ordinating
student participation in athletic, cultural, social, academic, and recreational
activities; encouraging cooperation and communication between students;
increasing awareness of the Faculty and of mathematics in general to the outside
community; by representing students to the Faculty and to the University; and by
providing students with the means to help them accomplish their goals both
within and without the University.

%% DEFINITIONS (d)
\section{Definitions}
In this and any other document of the Society, the following definitions shall
hold unless otherwise specified:
\begin{description}
\item[University]\hfill\\
  The University of Waterloo
\item[Faculty]\hfill\\
  The Faculty of Mathematics of the University
\item[Dean]\hfill\\
  The Dean of the Faculty
\item[Federation]\hfill\\
  The Federation of Students, University of Waterloo
\item[Council]\hfill\\
  The Society Council
\item[Executive]\hfill\\
  Any of the officers that form the Executive Committee
\item[member]\hfill\\
  Without further qualification, any social member of the Society
\item[academic year]\hfill\\
  Such a period of time as defined by the Senate of the University
\item[term]\hfill\\
  A period of time, approximately four months long, from the start of classes or
  of Orientation week in one term as defined by the Senate of the University,
  until the day before the start of next term
\item[Winter term]\hfill\\
  The term extending roughly from January to April of any given year
\item[Spring term]\hfill\\
  The term extending roughly from May to August of any given year.
\item[Fall term]\hfill\\
  The term extending roughly from September to December of any given year
\item[student]\hfill\\
  Any person registered as an undergraduate student at the University, including
  persons registered at another post-secondary institution for a joint program
  with the University
\item[math student]\hfill\\
  Any student registered in the Faculty, including in a joint program with
  another faculty or institution
\item[first-year student]\hfill\\
  Any student registered as being in their first year of study in the
  University, regardless of whether or not they are actually in their first year
  of study
\item[VPF]\hfill\\
  The Vice-President, Finances
\item[VPO]\hfill\\
  The Vice-President, Operations
\item[VPI]\hfill\\
  The Vice-President, Internal
\item[VPA]\hfill\\
  The Vice-President, Academic
\item[decision of the Society]\hfill\\
  A decision made by Council, a general meeting, a referendum, or any other body
  to which the power to make that decision was duly and legally delegated, or a
  document duly enacted by one of the above
\end{description}

With regards to any assembly within the Society, a non-voting member has all
rights accorded to regular members except for the right to make motions and
the right to vote.

Where someone is entitled to appoint a designate to serve a function
under this document, such an appointment shall be made by written notice to
the President or to the Speaker of Council, and notice of that appointment
shall be given to Council. More than one person may be so designated, but no
more than one person shall exercise the rights provided to a single person at
any given time.

%% MEMBERSHIP (m)
\section{Membership}
\subsection{Membership Fee}
The Society shall levy a membership fee to either be collected by the University
as a portion of student fees, or paid directly to the Society. Some members are
exempt from having to pay the membership fee, as defined elsewhere in this
document.

The amount of the MathSoc Fee may be adjusted only through one of the following two mechanisms:
\begin{enumerate}
\item Once per Fall term, by a resolution of Council, specifying an adjustment of a percentage equal
to or less than the increase in the Consumer Price Index for Canada in the previous calendar year
according to Statistics Canada. This increase is subject to ratification at the next General Meeting; or
\item Modified or removed by a referendum held in accordance with these bylaws.
\end{enumerate}

If a student has arranged fee payment to the satisfaction of the University and
the arranged fees include the fee for a given term, then that student is
considered to have paid the fee for that term, regardless of whether or not the
Society has received the funds.

Members of the Society have the right to request a fee refund within any
procedures set out by a decision of the Society. If they are not exempt from
paying the membership fee, then upon submitting a refund request, their rights
as members cease, but they are entitled to receive the refund only if they have
not in the interim used any services of the Society.

\subsection{Voting Membership}
The voting members of the society are those social members who meet one or more
of the following criteria:
\begin{itemize}
  \item A full-time or part-time math student in the current term.
  \item A math co-op student in the current term who was a voting member in the
    previous term.
  \item A voting member in the previous term who is slated to be a full-time or
    part-time math student in the next term and is not currently engaged in
    academic study.
\end{itemize}

\subsection{Social Membership}
The social members of the society are those students (undergraduate and
graduate), staff, faculty, or alumni at the University who have paid the Society
membership fee, as well as all full-time employees of the Society and all
Honorary Lifetime Members of the Society regardless of whether or not they have
paid the fee.

Additionally, if a person is not a student in the Faculty and would be a voting
member by the above section if they were a social member, then they are a social
member.

\subsection{Honorary Lifetime Members}
The Honorary Lifetime Members of the Society are those persons who have made
exceptionally significant contributions to the Society or towards its goals.
Honorary Lifetime Memberships may be conferred only by a three-quarters majority
vote, conducted by secret ballot, of a general meeting of the Society.

Honorary Lifetime Memberships are valid for the lifetime of the Society and
cannot be revoked. Honorary Lifetime Members cannot have obligations imposed on
them due to their status; if they accept a position within the Society, however,
they are still obligated to fulfill the duties of that position.

\subsection{Rights of Voting Members}
Voting members have the exclusive right to participate in Society
decision-making:
\begin{itemize}
  \item Vote at general meetings of the Society;
  \item Sign petitions of the Society;
  \item Vote in an election to any seat on Council, including the Executive, or
    in a referndum of the Society; and
  \item Nominate for, stand as candidate for, or a hold seat on Council,
    including the Executive.
  \item Inspect the financial records of the Society and, at their own expense,
    request a professional audit.
\end{itemize}

\subsection{Rights of Social Members}
Social members have, except where described otherwise in this document, the
right to pariticipate fully in activities in the Society, although this does not
mean that the Society cannot charge a fee for an event or that activities cannot
be limited to some subset of members, provided that all members are given fair
opportunity to be included.

Social members additionally have the right to participate in any general meeting
of the Society as non-voting members, and to view any governing documents or
public correspondance of the Society.

%%COUNCIL (c)
\section{Council}
\subsection{Composition}
The Society Council's is comprised of up to thirty-five voting members:
\begin{itemize}
  \item The Executive, defined in the following article; and
  \item Up to thirty Representatives, allocated as described below.
\end{itemize}

\noindent Additionally, there are non-voting members of Council:
\begin{itemize}
  \item The Speaker of Council;
  \item The Secretary of Council;
  \item The chairs of any committees or boards established by the Society or by
    Council;
  \item Any directors or executive assistants appointed by decision of the
    Society;
  \item The members of the Federation Orientation Committee representing the
    Faculty;
  \item The editors of \emph{math}{\sf NEWS};
  \item The undergraduate student Senators representing the Faculty and
    undergraduates at-large;
  \item The President of the University, or designate;
  \item The Dean of the Faculty, or designate;
  \item The Associate Dean, Undergraduate Studies of the Faculty, or designate;
  \item The Directors of the Federation or their respective designates;
  \item The representatives of the Faculty to the Students' Council of the
    Federation.
  \item The President of any society recognized by the Federation, or their
    respective designates; and
  \item The President of a club under the aegis of MathSoc.
\end{itemize}

If a person qualifies as both a voting and non-voting member of Council, then
they are a voting member.

\subsection{Representative Allocation}
Each Representative on Council shall represent a single constituency from the
following, determined by reference to the Faculty's undergraduate calendar. For
the purposes of determining if a student is in a program, minors and options are
not counted unless explicitly mentioned. A student with multiple plans may be
counted in more than one constituency. Notwithstanding anything else in this
section, if a student is counted in the First Year constituency, then they are
counted in no other constituency.

The constituencies are
\begin{itemize}
  \item First Year, consisting of all math students registered as first-year
    with the University, except for those in the Software Engineering program;
  \item Actuarial Science, consisting of all math students in Actuarial Science
    or Mathematical Finance programs;
  \item Statistics, consisting of all math students in Statistics programs;
  \item Pure Mathematics, Applied Mathematics, and Combinatorics and
    Optimization, consisting of all math students in Pure Mathematics, Applied
    Mathematics, Combinatorics and Optimization, Mathematical Finance, or
    Computational Mathematics programs;
  \item Computer Science, consisting of all math students in programs offered by
    the David R. Cheriton School of Computer Science, as well as all students in
    Computational Mathematics programs;
  \item Business, consisting of all math students in Mathematics/Business
    programs, as well as all students in the Business Administration and
    Computer Science Double Degree program;
  \item Computing and Financial Management, consisting of all math students in
    the Computing and Financial Management program;
  \item Software Engineering, consisting of all math students in the Software
    Engineering program;
  \item Teaching, consisting of all math students in the
    Mathematics/Teaching or Pure Mathematics/Teaching programs;
    and
  \item Mathematical Studies and Other, consisting of all math students in
    Mathematical Studies programs and all math students not counted in one of
    the other constituencies.
\end{itemize}

At the beginning of each Fall term, the President shall obtain enrollment
numbers from the University and determine the allocation of thirty
Representative seats to constituencies using the method of equal proportions
described in the appendix. The President shall present these to the general
meeting in that term to approve them for use in the subsequent year, except that
during the Spring term there will be no seats in the First Year constituency and
hence fewer than thirty seats.

\subsection{Duties \& Powers}
Council has full power over Society affairs except as such would conflict with
this document or a decision made at a general meeting or by a referendum.
Council is a fully-constituted assembly in its own right, and does not report to
general meetings, though it is accountable to them and to the members of the
Society at large.

Council can enact policies of the Society and direct the affairs of the Society,
except that Council has no power over the internal affairs of a general meeting
unless such power is otherwise delegated to it.

Council is responsible to uphold the purposes of the Society and to ensure that
the Society is not abused. It is responsible to hold the Executive and any other
persons involved in Society affairs to account, and the Representatives are
responsible for voicing the concerns and issues of their constituents and,
indeed, to represent them. To this end, Councillors shall maintain at least one
public office hour per week and shall inform their constituents of when they are
available. Executives shall maintain at least three office hours per week.

Individual Councillors are expected to attend Council meetings regularly. If a
Councillor misses or is more than one half-hour late for a meeting, that Council
member shall be deemed delinquent for that meeting. If a Councillor is
delinquent for three or more meetings in a given term, then that Councillor may
be removed from that or any other seat for the remainder of the year by majority
vote with notice.

\subsection{Convocation}
Meetings of Council may be called by any of the following;
\begin{itemize}
  \item The President;
  \item The Speaker of Council;
  \item Any three voting members of Council, upon petition in writing;
  \item Any twenty-five voting members of the Society, upon petition in writing;
    or
  \item The Dean or his designate.
\end{itemize}

During the period of classes in each term, Council shall meet no less than once
every three weeks.

\subsection{Notice}
Notice must be provided at least 48 hours in advance of any meeting to every
voting member of Council unless that member explicitly waives their right to
notice before the start of the meeting.

Where notice is required of a motion, notice of that motion must be provided at
least seven days in advance of the meeting at which the motion is to be
considered to every voting member of Council unless that member explicitly
waives their right to notice before the start of the meeting at which the motion
is moved. A full description of the intended motion, such as the text of a
proposed amendment or agreement, must be provided, but the motion may be
amended before or after it is moved as long as the changes remain within the
scope of the motion for which notice was given.

Council may designate a mailing list or similar forum to be the official notice
forum of Council; if this is done, then any notice sent to that forum is
considered to have been sent to every voting member of Council regardless of
whether or not it was received by that member.

\subsection{Elections \& Terms of Office}
When multiple elections are held simultaneously, a voter in multiple
constituencies shall cast a vote for Representatives of only one constituency in
any election.

Councillors or Councillors-elect can resign by written submission to the rest of
Council. In the event that one or more seats become vacant during the first two
months of a term, or if a seat is left vacant after an election, a by-election
shall be held for those seats. If a seat is vacant for a future term, then a
by-election may be held before the start of the vacant seat’s term. If no
by-election is held by the start of that term, one shall be held as soon as
possible after the term begins.

All aspects of election procedure not defined explicitly in this document may be
set by a decision of the Society.

\subsubsection{Elected Executive \& Upper-Year Reps}
For the Executive other than the VPF and the Representatives other than First
Year and Software Engineering Representatives, general elections shall be held
in the fall term for each of the three terms of the subsequent year. In an
election, a candidate does not need to run for each of the three terms for which
the election is held. A single ballot shall be taken from each voter for all
three terms in conjunction, and the results shall be evaluated separately for
each term, but a candidate cannot win a term for which they are not running.
Elections shall be conducted using a preferential ballot system.

\subsubsection{Vice-President, Finances}
The Vice-President, Finances, shall be appointed for each term by Council in the
previous term.

\subsubsection{Software Engineering Representatives}
At the start of each term, the students in each on-stream Software Engineering
class shall elect members of their class to serve as Software Engineering
Representatives for the term, in accordance with the usual process of election
of class representatives in the Faculty of Engineering. The available seats
shall be divided evenly between the classes to elect, with any extra seats going
first to lower-year classes. In the Spring term, seats shall be allocated as if
there is a first-year class, but the first-year seats shall remain vacant.

\subsubsection{First Year Representatives}
First Year Representatives shall be elected between (inclusively) September 15th and October 15th of the same year, who shall serve until the end of the next Winter term. A First-Year Representative does not lose their seat due to redistribution of seats at the start of Winter term. Elections shall be conducted using a preferential ballot system.

\subsection{Eligibility Requirements}
In order to run for or serve as a Representative, a member must be in the
constituency of their seat or show proof that they intend to register in their
constituency in their term of office.

In order to serve as an Executive, a member must be a math student, with no
other full-time non-academic commitments, other than a first-year student, and
be 18 years of age or older.

If a Councillor or Councillor-elect fails to meet these requirements, then they
do not lose their seat automatically, but may be removed from their seat by a
majority vote of Council.

No member shall occupy more than one voting seat on Council in the same term
simultaneously, but a member may run simultaneously for one Executive seat and
one Representative seat in the same term, and member in a Representative seat in
a given term may run in a by-election for an Executive seat in that same term.
If a member is elected to both an Executive and a Representative seat
simultaneously, then they take the Executive seat and the votes cast in the
Representative election are reevaluated with the member disqualified. If a
Representative is elected to an Executive seat, then they lose their
Representative seat.

\subsection{Quorum}
Two-fifths of all voting members of Council shall constitute a quorum.

\subsection{Sessions}
For greater certainty, each meeting as called in accordance with this document,
plus its adjournments, constitute a single session of Council.

\subsection{Speaker \& Secretary}
The Speaker of Council and the Secretary of Council shall be appointed by
Council. If there is no Speaker and/or no Secretary, then Council may not proceed to
any other business until a Speaker and/or a Secretary is appointed.  At the end of
each term, the Speaker and Secretary are automatically resigned, although they
may resign or be replaced by Council at an earlier time. At any time when there
is no Speaker and/or no Secretary, the President shall have the powers and
duties of the Speaker and/or Secretary, as appropriate.

\subsubsection{Duties of the Speaker}
The Speaker has the following duties:
\begin{itemize}
  \item Serve as the presiding officer of Council;
  \item Interpret this and any other governing documents of the Society, subject
    to appeal to Council or a general meeting;
  \item Arrange for and advertise meetings of Council;
  \item Ensure that all Council members have access to the official notice
    forum, if any; and
  \item Ensure that Council meetings are called regularly.
\end{itemize}

\subsubsection{Duties of the Secretary}
The Secretary has the following duties:
\begin{itemize}
  \item Serve as the secretary of Council;
  \item Record attendance of Council meetings, including when a member is more
    than one half-hour late for a meeting and if they sent notice of their
    absence;
  \item Distribute minutes of each Council or general meeting in a timely manner
    after that meeting;
  \item Absent a decision to the contrary by the assembly, serve as the
    secretary of general meetings;
  \item Report to Council when a Councillor is failing to meet the requirements
    of office; and
  \item Maintain and update the official copies of this document and any other
    vital documents of the Society, including internal policies and external
    agreements.
\end{itemize}

%%EXECUTIVE (x)
\section{Executive}
\subsection{Composition}
The Executive Board of the Society shall be composed of five officers: the
President; the Vice-President, Finances; the Vice-President, Operations; the
Vice-President, Internal; and the Vice-President, Academic.

\subsection{Duties and Powers}
The Executive are responsible for performing all duties assigned to them by a
decision of the Society, and are accountable to Council and to every member of
the Society (including, but not limited to, at a general meeting) for the
actions they take in performing their duties and exercising their power.

The Executive are responsible for generally maintaining the affairs of the
Society between meetings of Council, making recommendations to Council for
action, and ensuring that decisions of the Society are implemented. No action of
the Executive shall conflict with any decision of the Society.

No Executive shall approve funding to any organization in which they hold an
executive or similar position, unless such funding has explicitly been approved
by a decision of the Society.

All of the Executive are expected to attend regular meetings with representatives of
the Dean’s office.

The Executive Board is expected to send out regular communications to students in
the constituency.

\subsubsection{President}
The President is the chief executive of the Society, and shall:
\begin{itemize}
  \item Arrange for, advertise, and, absent a decision of the assembly to the
    contrary, chair general meetings;
  \item Represent the Society at official functions and public occasions;
  \item Serve as an ex-officio member of all committees and boards of the
    Society, except for committees whose purpose is primarily to nominate or
    recommend persons for an award or office;
  \item Verify the validity of petitions of the Society;
  \item Work with external organizations on behalf of the Society;
  \item Where not provided otherwise by a decision of the Society, make
    appointments of members to external bodies on behalf of the Society, and in
    any case communicate the appointments to those bodies;
  \item Represent the Society and its members to other organizations;
  \item Attend meetings of the Feds Committee of Presidents;
  \item Work with the other Executive to ensure that the transition from one
    term to the next goes smoothly; and
  \item Oversee the Math C\&D and the C\&D Manager
\end{itemize}

For greater certainty, the President need not seek election to external bodies
in order to satisfy the requirement that he represent the Society and its
members.

\subsubsection{Vice-President, Finances}
The Vice-President, Finances is responsible for the financial affairs of the
Society and shall:
\begin{itemize}
  \item Keep accurate and complete records of the finances of the Society;
  \item Prepare a budget, an opening financial report, and a closing financial
    report for the Society for each term and present them to Council;
  \item Present up-to-date financial reports to termly general meetings.
  \item Manage the accounts and funds of the Society;
  \item Within two weeks of a request, present the financial records of the
    Society to any member; and
  \item As soon as possible at the start of the term, check the accuracy and
    consistency of the previous term's financial records.
\end{itemize}

\subsubsection{Vice-President, Operations}
The Vice-President, Operations is responsible for the day-to-day operations of
the Society and shall:
\begin{itemize}
  \item Oversee and manage the services operated by the Society;
  \item Oversee and manage the Society Office;
  \item Allocate and manage the use of any space allocated to the Society; and
  \item In conjunction with the other Executives, arrange suppliers for the
    Society and ensure that the Society is stocked in any supplies it needs.
\end{itemize}

\subsubsection{Vice-President, Internal}
The Vice-President, Internal is responsible for overseeing Society events and
shall:
\begin{itemize}
  \item Serve as the final approver for all Society events, ensuring that all Society events have the appropriate documentation,
    including insurance coverage and/or event forms;
  \item Encourage members to become more involved in the Society and ensure that
    the opportunity exists for them to do so;
  \item Oversee all volunteers of the Society, including selection of directors;
    \item Oversee all internal organizations on behalf of the society, including Clubs and Affiliates, ensuring they are supported in their endeavours and compliant with MathSoc policy and financial/accounting requirements;
  \item Meet with every club/service executive at least once each term;
  \item Ensure that volunteers within the Society are appropriately recognized for their efforts; and
  \item Ensure that the spirit of Math does not leave the Society.

\end{itemize}

\subsubsection{Vice-President, Academic}
The Vice-President, Academic is reponsible for academic operations of the
Society and shall:
\begin{itemize}
  \item Represent the Society and its members to the Faculty, to the University,
    and to the Federation of Students on academic issues;
  \item Ensure that members have access to up-to-date academic information;
  \item When changes are made to the programs offered by the Faculty, if
    necessary, suggest changes to the way that Council seats are allocated to
    accommodate the changes.
  \item In conjuction with the other Executives, organize events and manage
    services of an academic nature.
\end{itemize}

For greater certainty, the Vice-President, Academic need not seek election to
external bodies in order to satisfy the requirement that he represent the
Society and its members.

\subsection{Incapacitation}
In the event that an Executive becomes unable to fulfill their duties, then
three voting members of Council may, with the written approval of the Dean or
his designate, declare that Executive to be incapacitated. An Executive may also
declare themselves to be incapacitated by written submission to the rest of
Council.

If an Executive, including the President, is declared incapacitated or ceases to
hold their position for any reason, a decision of the Society shall appoint an
interim replacement. If it is necessary for Council to make the appointment,
Council shall convene as soon as is practicable to do so, and the remaining
Executive, if any, shall recommend a potential appointee at that meeting. If one
voting member of Council is appointed as interim replacement for an Executive,
then for the duration of the time that they serve as interim replacement, their
original powers and duties shall be suspended and, if the replacement is
themselves an Executive, an interim replacement shall be appointed for them and
so on. For greater certainty, the duties and powers referred to in this
paragraph include those of being voting member of Council, as is inherent in
each of the Executive positions.

An Executive may resume their duties by providing written notice to Council at
least three days in advance. A notice by an Executive declaring themselves to be
incapacitated for a fixed period of time counts as notice for this purpose. At
the date specified in the notice, unless indicated otherwise by an intervening
written notice from that Executive, they resume their duties and powers. If
their interim replacement was another Executive, then that Executive resumes
their duties at the same time with no requirement of notice on their part.

\subsection{Remuneration of Executives}
As a form of compensation, each Executive will receive an honorarium of \$300 for serving in a given academic term, provided that they have been in the role for the majority of the term, are still in their role on the last day of the term. This honorarium will be awarded through a single payment payment within two weeks of the report on their performance being presented to a General Meeting. This does not preclude other non-monetary compensation. 

In the event that an Executive fails to effectively perform his or her duties, a resolution by a General Meeting of the Society may strip them of their honorarium with a two-thirds vote, provided that the meeting occurs before the payment is awarded. Debate on any such resolution will be held in secret session, without the presence of the Executive(s) in question. Furthermore, if a voting member so desires at the General Meeting, discussion of the Executive(s)' evaluation report (as presented by the Executive Evaluation Committee) will also be held in secret session, without the Executive(s) in question.

%%GENERAL MEETINGS(g)
\section{General Meetings}
\subsection{Convocation}
General meetings of the Society may be called by any of the following:
\begin{itemize}
  \item The President;
  \item Council;
  \item A general meeting; or
  \item Any one hundred voting members, upon petition in writing;
\end{itemize}

A general meeting shall be held in the third month of each term. If, by the
start of the month in which a meeting is to be held, the President has not
made public the date of the meeting, then any member may call the meeting, the
above notwithstanding.

\subsection{Notice \& Agenda}

Notice shall be delivered to the voting members of the society via their official university email no less than 10 business days before the meeting.

The complete agenda, including the full text of any motions for which notice shall be required, shall be delivered to the voting members of the society via their official university email no less than 5 business days before the meeting.

Posters detailing the time, place, and tentative agenda shall be posted in physical and visible locations within the Faculty buildings (currently DC, MC, M3) no less than 5 business days before the meeting.

The society should endeavour to publish notice of a meeting in the appropriate student publications.

Prior to five business days to the meeting, items can be added to the agenda by the President, by Council or by any twenty five voting member, upon petition in writing.

\subsection{Members}
All voting members are entitled to participate at general meetings. All other
members of the society are entitled to participate in a non-voting capacity.

\subsection{Proxies}
Each member is entitled to designate anyone as a proxy to participate in his
place at a general meeting. No person may be proxy for more than one principal
at the same meeting. Proxies shall be submitted in writing to the President at
least 24 hours prior to the start of the meeting.

The principal's rights as a member, voting or non-voting as the case may be, are
transferred to the proxy and the proxy possesses them indepedently of any rights
they may already have as a member. This may entitle them to two votes or double
the usual speaking time. A proxy shall use the rights conferred in this fashion
as directed by the principal.

\subsection{Quorum}
25 votes, counting proxies, constitute a quorum.

\subsection{Powers}
A general meeting has full power over the Society and its affairs, except as
limited by this document or by a decision made by referendum or at a general
meeting.

For greater certainty, a general meeting can exercise any power that Council can
exercise, including powers related to Council's internal affairs.

%%REFERENDA (r)
\section{Referenda}
\subsection{Initiation}
From time to time, the Society may consult its voting members by the means of a
referendum.

A referendum may be called by any of the following:
\begin{itemize}
  \item The President;
  \item Council;
  \item A general meeting; or
  \item Any one hundred fifty voting members of the Society, upon petition in
    writing.
\end{itemize}

The decision or petition requisitioning a referendum must include the full and
exact text of the question.

\subsection{Procedure}
Conspicuous and copious notice of at least seventy-two hours shall be given to
all voting members of the Society of the referendum, including those not
currently studying full-time.

The remainder of the referendum procedure shall be defined by decision of the
Society.

\subsection{Resolution}
The results of a referendum are fully binding on the Society until legally
overturned. For a period of eight years after a referendum, the referendum
cannot be overturned except by another referendum. After that, any decision of
the Society can overturn a referendum.

The members of the Society shall be informed of the results of a referendum as
soon as possible and in a manner similar to the way in which the notice was
given.

\subsection{Reconsideration}
A referendum on substantially the same question as one held in the past four
years shall be deemed to be a reconsideration of the earlier referendum.
Notwithstanding the above, a reconsideration can be called on and only on the
petition in writing of at least as many voting members of the Society as voted
for the winning outcome of the earlier referendum.

No reconsideration may be called within six months of the referendum it is to
reconsider without approval of Council by a two-thirds majority vote. No
reconsideration may be called within eighteen months of a prior reconsideration
of the same referendum.

\subsection{Recall}
A referendum may be called to remove a specific voting member and/or member-elect
from Council. Notwithstanding the above, such a referendum can be called on and
only on a petition in writing of at least one hundred fifty voting members, in
the case of an Executive position, or of at least twenty five voting members in
the appropriate consituency, in the case of a Representative position. In the
case of a person elected for multiple positions, the petition requirements stand
for each individual position.

If such a referendum passes, then its sole effect is to remove the person from
their position on Council and/or to cancel their election thereto. They are
eligible to run again in a by-election or any subsequent election. If re-elected
after being recalled, no further recall referendum may be called for that
position without approval of Council by a two-thirds majority vote.

%%COLLAPSE (l)
\section{Collapse \& Dissolution}
The Society shall be deemed to be collapsed if, for a period of at least seven
months, no quorate meeting of Council nor general meeting is held.

In the event of a collapse, then notwithstanding the rest of this document, any
voting member may call a general meeting, and this meeting has the power to
appoint persons directly to Council for the current term. It can remove people
from their office but it cannot appoint someone to an office for which they are
not eligible.

If no quorate meeting of Council nor general meeting is held for fourteen months, the
Society shall dissolve. Otherwise, the Society can dissolve only by referendum.
In the event that the Society dissolves, all of its assets shall go to the
Federation, to be used for the establishment of a new society for students of
the Faculty or to otherwise benefit the students of the Faculty.

%%PARLIAMENTARY AUTHORITY (u)
\section{Parliamentary Authority}
The Society shall use the most recent edition of
\emph{Robert's Rules of Order, Newly Revised} as its parliamentary authority,
and it is binding on the Society where not contradicted by this document or any
other decision made with due authority.

%%AMENDMENTS (a)
\section{Amendments}
Amendments to this document may be made by a referendum or by a two-thirds
majority vote of Council or of a general meeting, with notice.

If an amendment is made by Council, then it can take effect immediately (or as
specified in a proviso), but it shall be subject to confirmation by a general
meeting. A general meeting may reject or confirm the amendment - if rejected
or if not confirmed by the end of the subsequent regular termly general meeting,
then the amendment is reversed and no subsequent amendment by Council of similar
substance has any effect until confirmed at a general meeting. An amendment
passed at a general meeting implictly confirms similar amendments by Council.

\subsection{Contracts}
For the Society to enter into or assent to an amendment to a contract
that binds it for a duration of more than one term, it must be explicitly
approved by the same process as an amendment to this document, except that there
is no requirement of confirmation by a general meeting.

External agreements entered into by the Society must be compliant with this
document and any relevant decisions of the Society. Additionally, unless it
terminates sooner, the Society can enter into an agreement only if it contains a
clause requiring that it be reviewed by the parties to it no less often than
once every four years.

%%%APPENDICES
\appendix
\titleformat{\section}{\bfseries \LARGE \sffamily}
            {Appendix \thesection.}{1.6em}{}
%%METHOD OF EQUAL PROPORTIONS (p)
\section{Method of Equal Proportions}
To allocate the Representative seats on Council, first allocate to each
constituency a single seat. Then, for each consituency, its priority is $P =
\sqrt{\frac{c}{n(n+1)}}$, where $n$ is the number of seats already allocated to
that constituency, and $c$ is the number of constituents in that constituency.

Once the priorities have been calculated, the constituency with the highest
priority is allocated an additional seat, and its divisor and priority are
recalculated. This process is continued until all 30 seats have been allocated.

If multiple constituencies are tied for the highest priority, then they are all
allocated a seat simultaneously, unless that would bring the total number of
allocated seats above the maximum, in which case the tie will be broken in favor
of the constituents in the constituencies. If there is a further tie, then
previous terms' data shall be consulted in reverse chronological order until a
term is found in which there is no tie. In the exceptionally unlikely event that
the tie remains unbroken, it shall be broken in favor of the first constituency
listed in this document.
