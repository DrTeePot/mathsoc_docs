%% MEMBERSHIP (m)
\section{Membership}
\subsection{Membership Fee}
The Society shall levy a membership fee to either be collected by the University
as a portion of student fees, or paid directly to the Society. Some members are
exempt from having to pay the membership fee, as defined elsewhere in this
document.

The amount of the MathSoc Fee may be adjusted only through one of the following two mechanisms:
\begin{enumerate}
\item Once per Fall term, by a resolution of Council, specifying an adjustment of a percentage equal
to or less than the increase in the Consumer Price Index for Canada in the previous calendar year
according to Statistics Canada. This increase is subject to ratification at the next General Meeting; or
\item Modified or removed by a referendum held in accordance with these bylaws.
\end{enumerate}

If a student has arranged fee payment to the satisfaction of the University and
the arranged fees include the fee for a given term, then that student is
considered to have paid the fee for that term, regardless of whether or not the
Society has received the funds.

Members of the Society have the right to request a fee refund within any
procedures set out by a decision of the Society. If they are not exempt from
paying the membership fee, then upon submitting a refund request, their rights
as members cease, but they are entitled to receive the refund only if they have
not in the interim used any services of the Society.

\subsection{Voting Membership}
The voting members of the society are those social members who meet one or more
of the following criteria:
\begin{itemize}
  \item A full-time or part-time math student in the current term.
  \item A math co-op student in the current term who was a voting member in the
    previous term.
  \item A voting member in the previous term who is slated to be a full-time or
    part-time math student in the next term and is not currently engaged in
    academic study.
\end{itemize}

\subsection{Social Membership}
The social members of the society are those students (undergraduate and
graduate), staff, faculty, or alumni at the University who have paid the Society
membership fee, as well as all full-time employees of the Society and all
Honorary Lifetime Members of the Society regardless of whether or not they have
paid the fee.

Additionally, if a person is not a student in the Faculty and would be a voting
member by the above section if they were a social member, then they are a social
member.

\subsection{Honorary Lifetime Members}
The Honorary Lifetime Members of the Society are those persons who have made
exceptionally significant contributions to the Society or towards its goals.
Honorary Lifetime Memberships may be conferred only by a three-quarters majority
vote, conducted by secret ballot, of a general meeting of the Society.

Honorary Lifetime Memberships are valid for the lifetime of the Society and
cannot be revoked. Honorary Lifetime Members cannot have obligations imposed on
them due to their status; if they accept a position within the Society, however,
they are still obligated to fulfill the duties of that position.

\subsection{Rights of Voting Members}
Voting members have the exclusive right to participate in Society
decision-making:
\begin{itemize}
  \item Vote at general meetings of the Society;
  \item Sign petitions of the Society;
  \item Vote in an election to any seat on Council, including the Executive, or
    in a referndum of the Society; and
  \item Nominate for, stand as candidate for, or a hold seat on Council,
    including the Executive.
  \item Inspect the financial records of the Society and, at their own expense,
    request a professional audit.
\end{itemize}

\subsection{Rights of Social Members}
Social members have, except where described otherwise in this document, the
right to pariticipate fully in activities in the Society, although this does not
mean that the Society cannot charge a fee for an event or that activities cannot
be limited to some subset of members, provided that all members are given fair
opportunity to be included.

Social members additionally have the right to participate in any general meeting
of the Society as non-voting members, and to view any governing documents or
public correspondance of the Society.