%%COUNCIL (c)
\section{Council}
\subsection{Composition}
The Society Council's is comprised of up to thirty-five voting members:
\begin{itemize}
  \item The Executive, defined in the following article; and
  \item Up to thirty Representatives, allocated as described below.
\end{itemize}

\noindent Additionally, there are non-voting members of Council:
\begin{itemize}
  \item The Speaker of Council;
  \item The Secretary of Council;
  \item The chairs of any committees or boards established by the Society or by
    Council;
  \item Any directors or executive assistants appointed by decision of the
    Society;
  \item The members of the Federation Orientation Committee representing the
    Faculty;
  \item The editors of \emph{math}{\sf NEWS};
  \item The undergraduate student Senators representing the Faculty and
    undergraduates at-large;
  \item The President of the University, or designate;
  \item The Dean of the Faculty, or designate;
  \item The Associate Dean, Undergraduate Studies of the Faculty, or designate;
  \item The Directors of the Federation or their respective designates;
  \item The representatives of the Faculty to the Students' Council of the
    Federation.
  \item The President of any society recognized by the Federation, or their
    respective designates; and
  \item The President of a club under the aegis of MathSoc.
\end{itemize}

If a person qualifies as both a voting and non-voting member of Council, then
they are a voting member.

\subsection{Representative Allocation}
Each Representative on Council shall represent a single constituency from the
following, determined by reference to the Faculty's undergraduate calendar. For
the purposes of determining if a student is in a program, minors and options are
not counted unless explicitly mentioned. A student with multiple plans may be
counted in more than one constituency. Notwithstanding anything else in this
section, if a student is counted in the First Year constituency, then they are
counted in no other constituency.

The constituencies are
\begin{itemize}
  \item First Year, consisting of all math students registered as first-year
    with the University, except for those in the Software Engineering program;
  \item Actuarial Science, consisting of all math students in Actuarial Science
    or Mathematical Finance programs;
  \item Statistics, consisting of all math students in Statistics programs;
  \item Pure Mathematics, Applied Mathematics, and Combinatorics and
    Optimization, consisting of all math students in Pure Mathematics, Applied
    Mathematics, Combinatorics and Optimization, Mathematical Finance, or
    Computational Mathematics programs;
  \item Computer Science, consisting of all math students in programs offered by
    the David R. Cheriton School of Computer Science, as well as all students in
    Computational Mathematics programs;
  \item Business, consisting of all math students in Mathematics/Business
    programs, as well as all students in the Business Administration and
    Computer Science Double Degree program;
  \item Computing and Financial Management, consisting of all math students in
    the Computing and Financial Management program;
  \item Software Engineering, consisting of all math students in the Software
    Engineering program;
  \item Teaching, consisting of all math students in the
    Mathematics/Teaching or Pure Mathematics/Teaching programs;
    and
  \item Mathematical Studies and Other, consisting of all math students in
    Mathematical Studies programs and all math students not counted in one of
    the other constituencies.
\end{itemize}

At the beginning of each Fall term, the President shall obtain enrollment
numbers from the University and determine the allocation of thirty
Representative seats to constituencies using the method of equal proportions
described in the appendix. The President shall present these to the general
meeting in that term to approve them for use in the subsequent year, except that
during the Spring term there will be no seats in the First Year constituency and
hence fewer than thirty seats.

\subsection{Duties \& Powers}
Council has full power over Society affairs except as such would conflict with
this document or a decision made at a general meeting or by a referendum.
Council is a fully-constituted assembly in its own right, and does not report to
general meetings, though it is accountable to them and to the members of the
Society at large.

Council can enact policies of the Society and direct the affairs of the Society,
except that Council has no power over the internal affairs of a general meeting
unless such power is otherwise delegated to it.

Council is responsible to uphold the purposes of the Society and to ensure that
the Society is not abused. It is responsible to hold the Executive and any other
persons involved in Society affairs to account, and the Representatives are
responsible for voicing the concerns and issues of their constituents and,
indeed, to represent them. To this end, Councillors shall maintain at least one
public office hour per week and shall inform their constituents of when they are
available. Executives shall maintain at least three office hours per week.

Individual Councillors are expected to attend Council meetings regularly. If a
Councillor misses or is more than one half-hour late for a meeting, that Council
member shall be deemed delinquent for that meeting. If a Councillor is
delinquent for three or more meetings in a given term, then that Councillor may
be removed from that or any other seat for the remainder of the year by majority
vote with notice.

\subsection{Convocation}
Meetings of Council may be called by any of the following;
\begin{itemize}
  \item The President;
  \item The Speaker of Council;
  \item Any three voting members of Council, upon petition in writing;
  \item Any twenty-five voting members of the Society, upon petition in writing;
    or
  \item The Dean or his designate.
\end{itemize}

During the period of classes in each term, Council shall meet no less than once
every three weeks.

\subsection{Notice}
Notice must be provided at least 48 hours in advance of any meeting to every
voting member of Council unless that member explicitly waives their right to
notice before the start of the meeting.

Where notice is required of a motion, notice of that motion must be provided at
least seven days in advance of the meeting at which the motion is to be
considered to every voting member of Council unless that member explicitly
waives their right to notice before the start of the meeting at which the motion
is moved. A full description of the intended motion, such as the text of a
proposed amendment or agreement, must be provided, but the motion may be
amended before or after it is moved as long as the changes remain within the
scope of the motion for which notice was given.

Council may designate a mailing list or similar forum to be the official notice
forum of Council; if this is done, then any notice sent to that forum is
considered to have been sent to every voting member of Council regardless of
whether or not it was received by that member.

\subsection{Elections \& Terms of Office}
When multiple elections are held simultaneously, a voter in multiple
constituencies shall cast a vote for Representatives of only one constituency in
any election.

Councillors or Councillors-elect can resign by written submission to the rest of
Council. In the event that one or more seats become vacant during the first two
months of a term, or if a seat is left vacant after an election, a by-election
shall be held for those seats. If a seat is vacant for a future term, then a
by-election may be held before the start of the vacant seat’s term. If no
by-election is held by the start of that term, one shall be held as soon as
possible after the term begins.

All aspects of election procedure not defined explicitly in this document may be
set by a decision of the Society.

\subsubsection{Elected Executive \& Upper-Year Reps}
For the Executive other than the VPF and the Representatives other than First
Year and Software Engineering Representatives, general elections shall be held
in the fall term for each of the three terms of the subsequent year. In an
election, a candidate does not need to run for each of the three terms for which
the election is held. A single ballot shall be taken from each voter for all
three terms in conjunction, and the results shall be evaluated separately for
each term, but a candidate cannot win a term for which they are not running.
Elections shall be conducted using a preferential ballot system.

\subsubsection{Vice-President, Finances}
The Vice-President, Finances, shall be appointed for each term by Council in the
previous term.

\subsubsection{Software Engineering Representatives}
At the start of each term, the students in each on-stream Software Engineering
class shall elect members of their class to serve as Software Engineering
Representatives for the term, in accordance with the usual process of election
of class representatives in the Faculty of Engineering. The available seats
shall be divided evenly between the classes to elect, with any extra seats going
first to lower-year classes. In the Spring term, seats shall be allocated as if
there is a first-year class, but the first-year seats shall remain vacant.

\subsubsection{First Year Representatives}
First Year Representatives shall be elected between (inclusively) September 15th and October 15th of the same year, who shall serve until the end of the next Winter term. A First-Year Representative does not lose their seat due to redistribution of seats at the start of Winter term. Elections shall be conducted using a preferential ballot system.

\subsection{Eligibility Requirements}
In order to run for or serve as a Representative, a member must be in the
constituency of their seat or show proof that they intend to register in their
constituency in their term of office.

In order to serve as an Executive, a member must be a math student, with no
other full-time non-academic commitments, other than a first-year student, and
be 18 years of age or older.

If a Councillor or Councillor-elect fails to meet these requirements, then they
do not lose their seat automatically, but may be removed from their seat by a
majority vote of Council.

No member shall occupy more than one voting seat on Council in the same term
simultaneously, but a member may run simultaneously for one Executive seat and
one Representative seat in the same term, and member in a Representative seat in
a given term may run in a by-election for an Executive seat in that same term.
If a member is elected to both an Executive and a Representative seat
simultaneously, then they take the Executive seat and the votes cast in the
Representative election are reevaluated with the member disqualified. If a
Representative is elected to an Executive seat, then they lose their
Representative seat.

\subsection{Quorum}
Two-fifths of all voting members of Council shall constitute a quorum.

\subsection{Sessions}
For greater certainty, each meeting as called in accordance with this document,
plus its adjournments, constitute a single session of Council.

\subsection{Speaker \& Secretary}
The Speaker of Council and the Secretary of Council shall be appointed by
Council. If there is no Speaker and/or no Secretary, then Council may not proceed to
any other business until a Speaker and/or a Secretary is appointed.  At the end of
each term, the Speaker and Secretary are automatically resigned, although they
may resign or be replaced by Council at an earlier time. At any time when there
is no Speaker and/or no Secretary, the President shall have the powers and
duties of the Speaker and/or Secretary, as appropriate.

\subsubsection{Duties of the Speaker}
The Speaker has the following duties:
\begin{itemize}
  \item Serve as the presiding officer of Council;
  \item Interpret this and any other governing documents of the Society, subject
    to appeal to Council or a general meeting;
  \item Arrange for and advertise meetings of Council;
  \item Ensure that all Council members have access to the official notice
    forum, if any; and
  \item Ensure that Council meetings are called regularly.
\end{itemize}

\subsubsection{Duties of the Secretary}
The Secretary has the following duties:
\begin{itemize}
  \item Serve as the secretary of Council;
  \item Record attendance of Council meetings, including when a member is more
    than one half-hour late for a meeting and if they sent notice of their
    absence;
  \item Distribute minutes of each Council or general meeting in a timely manner
    after that meeting;
  \item Absent a decision to the contrary by the assembly, serve as the
    secretary of general meetings;
  \item Report to Council when a Councillor is failing to meet the requirements
    of office; and
  \item Maintain and update the official copies of this document and any other
    vital documents of the Society, including internal policies and external
    agreements.
\end{itemize}