%%REFERENDA (r)
\section{Referenda}
\subsection{Initiation}
From time to time, the Society may consult its voting members by the means of a
referendum.

A referendum may be called by any of the following:
\begin{itemize}
  \item The President;
  \item Council;
  \item A general meeting; or
  \item Any one hundred fifty voting members of the Society, upon petition in
    writing.
\end{itemize}

The decision or petition requisitioning a referendum must include the full and
exact text of the question.

\subsection{Procedure}
Conspicuous and copious notice of at least seventy-two hours shall be given to
all voting members of the Society of the referendum, including those not
currently studying full-time.

The remainder of the referendum procedure shall be defined by decision of the
Society.

\subsection{Resolution}
The results of a referendum are fully binding on the Society until legally
overturned. For a period of eight years after a referendum, the referendum
cannot be overturned except by another referendum. After that, any decision of
the Society can overturn a referendum.

The members of the Society shall be informed of the results of a referendum as
soon as possible and in a manner similar to the way in which the notice was
given.

\subsection{Reconsideration}
A referendum on substantially the same question as one held in the past four
years shall be deemed to be a reconsideration of the earlier referendum.
Notwithstanding the above, a reconsideration can be called on and only on the
petition in writing of at least as many voting members of the Society as voted
for the winning outcome of the earlier referendum.

No reconsideration may be called within six months of the referendum it is to
reconsider without approval of Council by a two-thirds majority vote. No
reconsideration may be called within eighteen months of a prior reconsideration
of the same referendum.

\subsection{Recall}
A referendum may be called to remove a specific voting member and/or member-elect
from Council. Notwithstanding the above, such a referendum can be called on and
only on a petition in writing of at least one hundred fifty voting members, in
the case of an Executive position, or of at least twenty five voting members in
the appropriate consituency, in the case of a Representative position. In the
case of a person elected for multiple positions, the petition requirements stand
for each individual position.

If such a referendum passes, then its sole effect is to remove the person from
their position on Council and/or to cancel their election thereto. They are
eligible to run again in a by-election or any subsequent election. If re-elected
after being recalled, no further recall referendum may be called for that
position without approval of Council by a two-thirds majority vote.