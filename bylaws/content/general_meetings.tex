%%GENERAL MEETINGS(g)
\section{General Meetings}
\subsection{Convocation}
General meetings of the Society may be called by any of the following:
\begin{itemize}
  \item The President;
  \item Council;
  \item A general meeting; or
  \item Any one hundred voting members, upon petition in writing;
\end{itemize}

A general meeting shall be held in the third month of each term. If, by the
start of the month in which a meeting is to be held, the President has not
made public the date of the meeting, then any member may call the meeting, the
above notwithstanding.

\subsection{Notice \& Agenda}

Notice shall be delivered to the voting members of the society via their official university email no less than 10 business days before the meeting.

The complete agenda, including the full text of any motions for which notice shall be required, shall be delivered to the voting members of the society via their official university email no less than 5 business days before the meeting.

Posters detailing the time, place, and tentative agenda shall be posted in physical and visible locations within the Faculty buildings (currently DC, MC, M3) no less than 5 business days before the meeting.

The society should endeavour to publish notice of a meeting in the appropriate student publications.

Prior to five business days to the meeting, items can be added to the agenda by the President, by Council or by any twenty five voting member, upon petition in writing.

\subsection{Members}
All voting members are entitled to participate at general meetings. All other
members of the society are entitled to participate in a non-voting capacity.

\subsection{Proxies}
Each member is entitled to designate anyone as a proxy to participate in his
place at a general meeting. No person may be proxy for more than one principal
at the same meeting. Proxies shall be submitted in writing to the President at
least 24 hours prior to the start of the meeting.

The principal's rights as a member, voting or non-voting as the case may be, are
transferred to the proxy and the proxy possesses them indepedently of any rights
they may already have as a member. This may entitle them to two votes or double
the usual speaking time. A proxy shall use the rights conferred in this fashion
as directed by the principal.

\subsection{Quorum}
25 votes, counting proxies, constitute a quorum.

\subsection{Powers}
A general meeting has full power over the Society and its affairs, except as
limited by this document or by a decision made by referendum or at a general
meeting.

For greater certainty, a general meeting can exercise any power that Council can
exercise, including powers related to Council's internal affairs.