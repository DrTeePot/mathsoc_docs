%%EXECUTIVE (x)
\section{Executive}
\subsection{Composition}
The Executive Board of the Society shall be composed of five officers: the
President; the Vice-President, Finances; the Vice-President, Operations; the
Vice-President, Internal; and the Vice-President, Academic.

\subsection{Duties and Powers}
The Executive are responsible for performing all duties assigned to them by a
decision of the Society, and are accountable to Council and to every member of
the Society (including, but not limited to, at a general meeting) for the
actions they take in performing their duties and exercising their power.

The Executive are responsible for generally maintaining the affairs of the
Society between meetings of Council, making recommendations to Council for
action, and ensuring that decisions of the Society are implemented. No action of
the Executive shall conflict with any decision of the Society.

No Executive shall approve funding to any organization in which they hold an
executive or similar position, unless such funding has explicitly been approved
by a decision of the Society.

All of the Executive are expected to attend regular meetings with representatives of
the Dean’s office.

The Executive Board is expected to send out regular communications to students in
the constituency.

\subsubsection{President}
The President is the chief executive of the Society, and shall:
\begin{itemize}
  \item Arrange for, advertise, and, absent a decision of the assembly to the
    contrary, chair general meetings;
  \item Represent the Society at official functions and public occasions;
  \item Serve as an ex-officio member of all committees and boards of the
    Society, except for committees whose purpose is primarily to nominate or
    recommend persons for an award or office;
  \item Verify the validity of petitions of the Society;
  \item Work with external organizations on behalf of the Society;
  \item Where not provided otherwise by a decision of the Society, make
    appointments of members to external bodies on behalf of the Society, and in
    any case communicate the appointments to those bodies;
  \item Represent the Society and its members to other organizations;
  \item Attend meetings of the Feds Committee of Presidents;
  \item Work with the other Executive to ensure that the transition from one
    term to the next goes smoothly; and
  \item Oversee the Math C\&D and the C\&D Manager
\end{itemize}

For greater certainty, the President need not seek election to external bodies
in order to satisfy the requirement that he represent the Society and its
members.

\subsubsection{Vice-President, Finances}
The Vice-President, Finances is responsible for the financial affairs of the
Society and shall:
\begin{itemize}
  \item Keep accurate and complete records of the finances of the Society;
  \item Prepare a budget, an opening financial report, and a closing financial
    report for the Society for each term and present them to Council;
  \item Present up-to-date financial reports to termly general meetings.
  \item Manage the accounts and funds of the Society;
  \item Within two weeks of a request, present the financial records of the
    Society to any member; and
  \item As soon as possible at the start of the term, check the accuracy and
    consistency of the previous term's financial records.
\end{itemize}

\subsubsection{Vice-President, Operations}
The Vice-President, Operations is responsible for the day-to-day operations of
the Society and shall:
\begin{itemize}
  \item Oversee and manage the services operated by the Society;
  \item Oversee and manage the Society Office;
  \item Allocate and manage the use of any space allocated to the Society; and
  \item In conjunction with the other Executives, arrange suppliers for the
    Society and ensure that the Society is stocked in any supplies it needs.
\end{itemize}

\subsubsection{Vice-President, Internal}
The Vice-President, Internal is responsible for overseeing Society events and
shall:
\begin{itemize}
  \item Serve as the final approver for all Society events, ensuring that all Society events have the appropriate documentation,
    including insurance coverage and/or event forms;
  \item Encourage members to become more involved in the Society and ensure that
    the opportunity exists for them to do so;
  \item Oversee all volunteers of the Society, including selection of directors;
    \item Oversee all internal organizations on behalf of the society, including Clubs and Affiliates, ensuring they are supported in their endeavours and compliant with MathSoc policy and financial/accounting requirements;
  \item Meet with every club/service executive at least once each term;
  \item Ensure that volunteers within the Society are appropriately recognized for their efforts; and
  \item Ensure that the spirit of Math does not leave the Society.

\end{itemize}

\subsubsection{Vice-President, Academic}
The Vice-President, Academic is reponsible for academic operations of the
Society and shall:
\begin{itemize}
  \item Represent the Society and its members to the Faculty, to the University,
    and to the Federation of Students on academic issues;
  \item Ensure that members have access to up-to-date academic information;
  \item When changes are made to the programs offered by the Faculty, if
    necessary, suggest changes to the way that Council seats are allocated to
    accommodate the changes.
  \item In conjuction with the other Executives, organize events and manage
    services of an academic nature.
\end{itemize}

For greater certainty, the Vice-President, Academic need not seek election to
external bodies in order to satisfy the requirement that he represent the
Society and its members.

\subsection{Incapacitation}
In the event that an Executive becomes unable to fulfill their duties, then
three voting members of Council may, with the written approval of the Dean or
his designate, declare that Executive to be incapacitated. An Executive may also
declare themselves to be incapacitated by written submission to the rest of
Council.

If an Executive, including the President, is declared incapacitated or ceases to
hold their position for any reason, a decision of the Society shall appoint an
interim replacement. If it is necessary for Council to make the appointment,
Council shall convene as soon as is practicable to do so, and the remaining
Executive, if any, shall recommend a potential appointee at that meeting. If one
voting member of Council is appointed as interim replacement for an Executive,
then for the duration of the time that they serve as interim replacement, their
original powers and duties shall be suspended and, if the replacement is
themselves an Executive, an interim replacement shall be appointed for them and
so on. For greater certainty, the duties and powers referred to in this
paragraph include those of being voting member of Council, as is inherent in
each of the Executive positions.

An Executive may resume their duties by providing written notice to Council at
least three days in advance. A notice by an Executive declaring themselves to be
incapacitated for a fixed period of time counts as notice for this purpose. At
the date specified in the notice, unless indicated otherwise by an intervening
written notice from that Executive, they resume their duties and powers. If
their interim replacement was another Executive, then that Executive resumes
their duties at the same time with no requirement of notice on their part.

\subsection{Remuneration of Executives}
As a form of compensation, each Executive will receive an honorarium of \$300 for serving in a given academic term, provided that they have been in the role for the majority of the term, are still in their role on the last day of the term. This honorarium will be awarded through a single payment payment within two weeks of the report on their performance being presented to a General Meeting. This does not preclude other non-monetary compensation. 

In the event that an Executive fails to effectively perform his or her duties, a resolution by a General Meeting of the Society may strip them of their honorarium with a two-thirds vote, provided that the meeting occurs before the payment is awarded. Debate on any such resolution will be held in secret session, without the presence of the Executive(s) in question. Furthermore, if a voting member so desires at the General Meeting, discussion of the Executive(s)' evaluation report (as presented by the Executive Evaluation Committee) will also be held in secret session, without the Executive(s) in question.