\documentclass[12pt, letterpaper]{mathsoc}
\usepackage{mou}

\orgOne{The Mathematics Society}
\orgTwo{The Mathematics Orientation Committee}
\signed{December 1, 2016}
\expires{November 30, 2017}

\begin{document}
\header

\section{Recognition of the Relationship}

\begin{itemize}
    \item Whereas the Mathematics Society is an independent organization responsible for serving and representing all undergraduate students in the Faculty of Mathematics; and
    \item Whereas the Math Orientation Committee is an organization reporting directly to the Office of the Dean of Mathematics, responsible for coordinating programs intended to introduce and welcome new students to the Faculty; and
    \item Whereas this committee is run by representatives of the Waterloo Orientation partnership, which includes the Federation of Students and the Student Success Office; and
    \item Whereas the committee is run exclusively by volunteers who are undergraduate Math students, and therefore members of the Mathematics Society; and
    \item Whereas the committee is not recognized by the Society as an Affiliate, or as an External Funding Group;
    \item This agreement recognizes that there exist many shared interests between the Society and the Committee; and
    \item This agreement further recognizes that there is a great deal of resource-sharing between both parties as a consequence of these overlapping interests; and
    \item This agreement further recognizes that this sharing of resources has resulted in an important relationship between both parties.
\end{itemize}

\section{Objectives of the Memorandum}

This agreement strives to:
\begin{enumerate}
    \item Define and clarify the relationship between the Mathematics Society and the Math Orientation Committee;
    \item Recognize that collaboration and shared resources will continue to exist between both parties, while:
    \begin{enumerate}
        \item Identifying the shared resources, and mutual areas of interest;
        \item Establishing a means by which each party will be able to hold the other accountable for the use of their resources;
    \end{enumerate}
    \item Outline the expectations of both parties with regards to:
    \begin{enumerate}
        \item Use of shared resources;
        \item Protection of mutual interests;
        \item Communication between both parties.
    \end{enumerate}
\end{enumerate}

\section{Agreement Between the Parties}

\subsection{The Mathematics Society}

The Mathematics Society agrees to:
\begin{enumerate}
    \item Provide an office space where the Math Orientation Committee can operate, with the following stipulations:
    \begin{enumerate}
    \item The office will be equipped with a phone;
    \item Only the MathFOC will be issued keys to the office, with the exception of the President of the Mathematics Society;
    \item Mail intended for Math Orientation will be delivered to the MathSoc main office;
    \end{enumerate}
    \item Accommodate Orientation’s unique needs with regards to venue bookings and uses;
    \item Provide Math Orientation with the necessary administrative tools to use their resources. Such resources include, but are not limited to:
    \begin{enumerate}
    \item Photocopier codes;
    \item Keys for office spaces;
    \item Mailbox space;
    \end{enumerate}
    \item Communicate with the Math Orientation Committee on any initiatives pertaining to new student engagement or transition, in order to minimize the risk of conflicting messages and or/ conflicting logistics;
    \item Assist the Committee with its annual volunteer recruitment through non-financial means, including (but not necessarily limited to):
    \begin{enumerate}
    \item Advertising via the Math Undergrad mailing list and/or MathSoc mailing lists;
    \item Providing the main office for use by Orientation during interviews (if necessary), outside of regular office hours;
    \end{enumerate}
\end{enumerate}

\subsection{The Math Orientation Committee}

The Math Orientation Committee agrees to:
\begin{enumerate}
    \item Use all shared resources appropriately, and in the manner for which they were intended;
    \item Pay repair/replacement costs for any damages incurred as a direct result of use or abuse by the Committee;
    \item Book space as needed in accordance with MathSoc’s booking policies;
    \item Provide MathSoc with pink ties to use for their own purposes each year, subject to the following stipulations:
\begin{enumerate}
    \item The design of the ties must remain the sole discretion of the Math Orientation Committee;
    \item The number of ties which MathSoc receives is at the discretion of the Committee, and will be decided based on the total ties left over after the Week, and Orientation’s projected need for further ties;
    \item At any time, Math Orientation may reclaim some or all of the ties, as their needs require, pursuant to the following stipulations: 
\begin{enumerate}
    \item The Committee must be able to demonstrate this need, if asked;
    \item The Committee must also ensure that the Office Manager and Vice President, Operations of the Mathematics Society is informed, so that the inventory can be adjusted accordingly;
\end{enumerate}
    \item For the first three weeks following Orientation, new students who attended Orientation, but who did not receive their pink tie will be directed to the Math Orientation Office;
\end{enumerate}
    \item Assist MathSoc with any Society initiatives pertaining to new student 
    engagement/transition, to minimize the risk of conflicting messages or 
    logistics between the two parties.
    \item Include MathSoc Day as an event in the orientation schedule, with a minimum duration of two (2) hours;
    \item Include a MathSoc branded item in the first year orientation kits;
    \item Defer to the Vice President, Operations, of the Mathematics Society on the appropriate use of MC 3038 (the MathSoc Main Office) during Orientation Week.
\end{enumerate}

\section{Handling of Disputes}

In the event that a dispute arises between both parties, and neither party can agree on a resolution to the dispute which is in keeping with this agreement, then a representative agreed upon by both parties shall intercede and mediate a solution.

\section{Expiry and Renewal of this Agreement}

This agreement shall expire on the date shown above. In the four months prior to the expiry of the agreement, both parties shall meet to review the agreement, and to discuss renewal, as well as any necessary alterations. This review and renewal process must be completed prior to the hiring of the FOC team for the subsequent year.

In the event that unforeseen circumstances result in either party being unable to continue fulfilling the terms of this agreement, at the consent of both parties the agreement may be opened for review and/or nullification at any time prior to its expiry.

\newpage
\section{Signatures}

By affixing their signatures below, the following individuals consent, on behalf of the parties they represent, to be held to the terms of this agreement as outlined above.

\signline{President,\\ Mathematics Society }

\signline{Speaker,\\ Mathematics Society Council}

\signline{FOC Representative,\\ Mathematics Orientation Committee}

\signline{Orientation Advisor,\\ Faculty of Mathematics}

\end{document}
