\asection{Name}

As far back as I have heard of, which is quite a ways into the past, there have
been a few constants around the third floor of the Math \& Computer building.
An unhealthy obsession for the colour pink emanating from the office in the
southeast corner. A strained relationship with the Computer Science Club across
the hall. A fornightly (not biweekly, please!) newspaper of questionable quality
and unquestionably lax editorial policies that redeems itself by publishing
funny quotations from professors, especially Larry Smith in economics.

And as far as I know, MathSoc has always called itself MathSoc. There was a
dabbling in the early 2000s (and possibly farther back) of two
mostly-independent societies, X, and Y, which alternated from term to term much
as EngSoc A and B do today. The intent was to make it easier for co-op students
(and anyone who looks at EngSoc's governance will understand why; it works very
well for them), but the nontrivial portion of regular students in the Faculty
makes for difficult work.

-Sean, Winter 2014

\asection{Object}

It's interesting to compare how the society nowadays works in relation to its
constitutional object. Since the Constitution \& Bylaws were entirely revised in
Winter 2011, the departure ought not to be too profound, but in some ways it is.
MathSoc's promotion of activities is frequently limited to the events it olds,
and increasing awareness to the outside community is also lacking---though a few
enterprising students are looking to change that. Representation has been a
strong point, if I do say so myself. I think that MathSoc could also do a lot
more to help the student body accomplish their goals.

On the flip side, a student society with a good object will always find itself
falling short, because its work is never done.

One thing worth noting is that this article is more than just philosophical.
This provides an important basis, both procedurally (see, for example,
\emph{RONR}, 11th. ed., p.~113, ll.~10-14) and politically, for the society to
frame its deliberations and actions. There are always arguments about what
MathSoc should or should not be doing, and this article can be powerfully
persuasive tool in such arguments, when it is applicable.

Also, I cannot understand for the life of me why I didn't format this as a
bulleted list. It's practically unreadable. Somebody should fix that.

-Sean, Winter 2014

\asection{Definitions}

This section, perhaps, has a few more definitions than are strictly necessary. A
few of the definitions deserver further comment, though.

The ``academic year'' as defined by the University is presently from September
to August. This is different from the fiscal year of all major organizations on
campus, which extends from May to April. This causes some difficulties, from
time to time. In practice, MathSoc takes a year to be whatever is convenient at
the time, because a year is a long time for undergrads.

The word ``term'' is highly overloaded, and this is unfortunate. It can mean
both the term of office of someone in a position, and of course the academic
term about 4 months long. Often you want to use both meanings in the same
sentence. There is no particularly good substitute in other case, because the
word ``tetramester'' hasn't really caught on. Often, the correct meaning needs
to be inferred from context.

The definition of a term here indirectly sets the term of office of Council and
of the Executive. This rarely matters, as good Executive tend to star early and
end late, but it does affect things like who gets invited to Orientation Week
events. It also would affect Council membership, should an emergency arise right
at a term boundary and action need to be taken quickly.

The definitions of ``student'' and ``math student'' are carefully written to be
as inclusive as possible. It seems that the University administration is always
inventing creative ways to blend programs across faculties and institutions,
such as the Double Degree programs with Laurier. Interestingly enough, this
does include students who are out of reach of UW at faraway institutions and who
are on exchange.

The definition of ``first-year student'' is similarly broad, and the intent here
is simply to ensure that when we get a list of first-years from the Registrar's
Office to put into our voter lists, it matches up with our definition of
first-year. The most notable case where a student is first-year but not in first
year is when a student fails enough courses that they fail to advance to 2A, as
your term of study in the Math Faculty is determined entirely by credits.

The final paragraphs of this article, relating to non-voting members and
appointments, are rules that exist just so that this case is covered, and not
because there is any particularly good reason for one version of a rule over
another. In particular, the fact that non-voting members cannot make motions is
rather arbitrary, and the rules for appointments just ensure that someone cannot
show up claiming to be a representative without authority (for my thoughts on
that, however, see the later section on non-voting members of Council!).

-Sean, Winter 2014

\asection{Membership}
The membership section is one of the most confusing, and probably could be
simplified. But it works and nobody really questions it. Still, rewriting this
section would be a good idea.

Basically, we can divide members into three categories:
\begin{enumerate}
  \item Voting members, who are math undergraduate students;
  \item Social but non-voting members, who pay the fee but aren't students; and
  \item Honourary Lifetime Members.
\end{enumerate}

There are some issues here though. So I'll come back to this.

\asubsection{Membership Fee}
foo
\asubsection{Voting Membership}
foo
\asubsection{Social Membership}
foo
\asubsection{Honorary Lifetime Members}
foo
\asubsection{Rights of Voting Members}
foo
\asubsection{Rights of Social Members}
foo

\asection{Council}
\asubsection{Composition}
foo
\asubsection{Representative Allocation}
foo
\asubsection{Duties \& Powers}
foo
\asubsection{Convocation}
foo
\asubsection{Notice}
foo
\asubsection{Elections \& Terms of Office}
foo
\asubsubsection{Elected Executive \& Upper-Year Reps}
foo
\asubsubsection{Vice-President, Finances}
foo
\asubsubsection{Software Engineering Representatives}
foo
\asubsubsection{First Year Representatives}
foo
\asubsection{Eligibility Requirements}
foo
\asubsection{Quorum}
foo
\asubsection{Sessions}
foo
\asubsection{Speaker \& Secretary}
foo
\asubsubsection{Duties of the Speaker}
foo
\asubsubsection{Duties of the Secretary}
foo

\asection{Executive}
\asubsection{Composition}
foo
\asubsection{Duties and Powers}
foo
\asubsubsection{President}
foo
\asubsubsection{Vice-President, Finances}
foo
\asubsubsection{Vice-President, Operations}
foo
\asubsubsection{Vice-President, Events}
foo
\asubsubsection{Vice-President, Academic}
foo
\asubsection{Incapacitation}
foo

\asection{General Meetings}
\asubsection{Convocation}
foo
\asubsection{Notice \& Agenda}
foo
\asubsection{Members}
foo
\asubsection{Proxies}
foo
\asubsection{Quorum}
foo
\asubsection{Powers}
foo

\asection{Referenda}
\asubsection{Initiation}
foo
\asubsection{Procedure}
foo
\asubsection{Resolution}
foo
\asubsection{Reconsideration}
foo
\asubsection{Recall}
foo

\asection{Collapse \& Dissolution}
foo

\asection{Parliamentary Authority}
foo

\asection{Amendments}
foo
\asubsection{Contracts}
foo

\appendix
\asection{Method of Equal Proportions}
foo

