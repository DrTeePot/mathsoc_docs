\section{Central Budget}
Effective March 6, 2007; November 27, 2006

\subsection{Approval of the Budget}
\begin{itemize}
\item The Vice-President, Finances shall present a central budget to Council for approval by the end of the first month of each term. The central budget shall include a statement of accounts held by the society.
\item All budget requests shall be submitted to the Vice-President, Finances, no later than the third week of the term.
\item If funding from a previous term was allocated but has not yet been spent, it is forfeit unless mentioned in a budget request for the current term. Council shall not refuse funding on a carry-over item that was allocated in either of the previous two terms, but may choose to reallocate the funds afterwards.
\end{itemize}


\subsection{Funding Carry-Over}
\begin{itemize}
\item If funding from a previous term was allocated but has not yet been spent, it is forfeit unless mentioned in a budget request for the current term. Council shall not refuse funding on a carry-over item that was allocated in either of the previous two terms, but may choose to reallocate the funds afterwards.
\item Funds in the Society's main accounts shall be included in the budget in each term as available funds, except for expenses that have been approved but that have not yet been reimbursed.
\item \$10,000 shall be maintained in the Society's main accounts as a float.
\end{itemize}

\subsection{Approval of Expenses}
\begin{itemize}
\item All non-budgeted expenses must be approved by members of the Executive Board or the Society Council as follows.
\begin{itemize}
\item For expenses under \$20.00, the approval of a single member of the Executive Board is required;
\item For expenses between \$20.00 and \$100.00, the approval of two members of the Executive Board, one of which must be the Vice-President, Finances, is required.
\item Council must approve all expenditures over \$100.00.
\end{itemize}
\item Budgeted expenditures must be approved by the appropriate member of the Executive Board. The Vice-President, Finances, must be informed of the approval as soon as possible. They shall not exceed more than 5\% of the budgeted amount without approval from Council.
\item The Vice-President, Finances shall report expenditures over the budgeted amount to Council.
\item No expense shall be approved from a previous term's budget after the budget meeting for a term unless the unallocated funding was reported and approved in the central budget in the current term.
\end{itemize}

\subsection{Reimbursement}
\begin{itemize}
\item Expenses incurred in the course of organizing, planning, and executing items of business for the Society are recoverable as long as the conditions in policies are met.
\item All expense requests must be accompanied by a receipt to be approved.
\item A record of expenditures to be reimbursed are to be submitted to the Executive Board within one week following the event. If exact values for the event are not known, an upper estimate should be provided immediately, and an event summary will be provided with appropriate figures as soon as possible. If this is not complied with the expenditure may not be reimbursed.
\end{itemize}

\subsection{Income}
\begin{itemize}
\item Income earned in the course of executing society business shall be counted and recorded on an appropriate income form and submitted to the VPF.
\end{itemize}

\subsection{Appropriate Use of Funds}
\begin{itemize}
\item Funds collected or managed by the Society or any organization directly responsible to the Society may not be used to purchase alcoholic beverages of any kind.
\end{itemize}