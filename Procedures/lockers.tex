\section{Locker Distribution}
effective July 17, 2014; replaces December 4, 2002; replaces March 23, 1999\\

Each member of the Society may use one locker per term, subject to availability. Additionally, up to 10\% of the lockers may be used by non-Society members, only available for registration after the first three weeks of the start of classes.

\subsection{Locker Booking}
These lockers shall be available for registration on the MathSoc website and the booking system shall remain open from the second week of class until all lockers have been occupied.\\
Upon request for a locker, the member will be randomly assigned a locker number and the appropriate combination. Once being issued the combination for the locker, the member may no longer change or revoke their own booking for the term. The booker also agrees to be the sole occupier of the locker at all times.

\subsection{Locker Administration}
No later than the beginning of the second week of classes each term, the Vice President, Operations shall be tasked with resetting the lockers. This entails:
\begin{enumerate}
\item Changing each locker combination in accordance with privacy and security guidelines.
\item Removing the contents of each locker, storing and labelling the contents and placing them in the MathSoc Office.
\end{enumerate}
Any contents that have been removed from a locker at any point during the time will be labelled and placed in the MathSoc Office for pickup. Belongings shall not be returned to students without photo identification and will be kept for a period of 30 days. After this period, all contents will be placed in the lost and found.

\subsection{Locker Use}
Lockers shall not be used to store items forbidden by law or University policy, including but not limited to: weapons, any flammable substances, explosive devices, or illegal substances. Lockers should also not be used to store food, drink or other perishable items.\\
MathSoc reserves the right to open any locker thought to be in violation of the above rule with 24 hours prior notice to the occupier.\\
MathSoc may remove only items that violate University policy or food, drinks, and other perishable items. MathSoc may dispose only of food, drinks, and other perishable items.
