\documentclass[11pt]{mathsoc}
\usepackage{constitution}

\title{Actuarial Science Club Constitution}

\begin{document}
%%% TITLE PAGE
%\mathsoctitle

%%% TABLE OF CONTENTS
\setcounter{tocdepth}{4}
{\large \tableofcontents}

\newpage

\section{Name}
The full name of the Club shall be “The University of Waterloo Actuarial 
Science Club”; its short name shall be “ActSci”.

\section{Purpose}
The University of Waterloo Actuarial Science Club (herein the Club) primary 
aims to increase students’ awareness and knowledge of the current actuarial 
profession, as well as to enhance the interaction amongst actuarial science 
students within the Faculty of Mathematics. The Club also maintains a library 
where students can borrow study material to assist in their studying of the 
professional actuarial exams.  This Club will serve as an important source for 
students to learn more about the field of actuarial science.  

\section{Membership}
There are two levels of membership Full Member and Associate Member.

Full member
Any Social member of MathSoc (Individual who pays their MathSoc Fee) of the 
mathematics society at the University of Waterloo can be a full member of the 
Actuarial Science Club if they pay their membership fee. A full member can 
vote in elections, hold an executive positions and attend events.

Associate Member
Any person who pays their membership fee and is not a full member is an 
associate  member of the Actuarial Science Club.  An Associate member can 
attend events.  

Membership Fee: 5 dollars

\section{Executive Positions}
President\\
VP Events (2 mentorship, and 3 event coordinators report to the VP Events)\\
VP Finance\\
VP Communication (1 newsletter editor, and 2 publicity directors report to the 
VP Communication)\\
VP Administration (1 website manager report to the VP Administration)\\

\section{Description of Positions}
President (1)\\
Represents the club to the Mathematics Society, and to all external 
organizations. The President also oversees all the vice-presidents. 
The President also has signing authority for the cheques. Must have 
two prior experiences with the club including one term as a Vice-President.

VP of Events (1)\\
Oversees the Event coordinators and Mentorship coordinators. Ensures 
that all events and mentorship events run smoothly. 

Must have 1 prior experience in the club.

VP of Finance (1)\\
Responsible for budgeting, reimbursements and is a signing authority. 

Must have 1 prior experience with the club.

VP of Communication (1)\\
Oversees the newsletter editor and publicity directors. Ensures that 
members are made aware of events through posters, Facebook events, and email. 

Must have 1 prior experience in the club.

VP of Administration (1)\\
Oversees the website manager. Ensures that the office is clean and manages the 
manuals. Responsible for seeking funding for new manuals or new initiative to 
help students with professional actuarial exams. 

Must have 1 prior experience with the club.

Event Coordinator (3)\\
Plans, runs, and coordinates all club events, and collaborates on mentorship 
events with the Mentorship Coordinators.

Mentorship Coordinator (2)\\
Plans, runs, and coordinates all club mentorship events in collaboration with 
Event Coordinators.

Newsletter Editor (1)\\
Ensures that all newsletters are published, puts together the newsletters.

Publicity Director (2)\\
Responsible for creating posters and Facebook events, and emailing students. 
Ensure that members are notified of events and services offered by the club.

Website Manager (1)\\
In charge of managing the website and making sure the computer keeps on working.

\section{Executive Selection Committee}
Elections will be held by the most senior person, who is on campus from the two
previous terms and not interested in the President position. Henceforth known 
as the Chief Returning Officer (CRO). The CRO will put up posters in the first 
of week of classes of each term, and the election will happen in the second week. 

The elections is to be in the format of two rounds of interviews, the first 
being to select the President and VPs, and the second to select the remaining
executives. If an interviewer is running for any of the positions, then they
are not able to interview candidates for the position(s). Any unfilled 
positions can be filled at the discretion of the executive team once the 
selection process has been completed.

The first round will have the following interviewers: 
The CRO from the two previous terms, as described above.
The Presidents and VPs from all four previous terms, who are on campus.
In the first round of interviews, the interviewers will select the President and 
VPs. A simple majority will be required to select the executive member. 
In the case of a 50--50 divide, the CRO will have the right to select the 
executive member.

The second round will have the following interviewers: 
The CRO from the two previous terms, as described above.
The Presidents from all four previous terms, who are on campus.
The President and VPs from the current term, selected from the first
round of interviews.
In the second round of interviews, the interviewers will interview the 
remaining non-President and non-VP candidates. A simple majority will 
be required to select the executive member. In the case of a 50--50 divide, the
CRO will have the right to select the executive member.

\section{Meetings}
A Quorum for a meeting is 20 members or 50\% of members whichever is lower.

\section{Impeachment}
To call a meeting to remove the President or a Vice-President, 25\% of the 
members or each of the other executives must sign a petition of removal. 
One week notice must be given to members of the meeting. At the meeting a 
majority of members present must vote by secret ballot for the impeachment 
of the executive.

To call a meeting to remove other executives, the President, and Vice-President
who is being reported must agree to call the meeting.

If an impeachment occurs notify all other executives immediately and all 
members via e-mail. 

\section{Ammendments}
To amend the constitution there needs to be a majority vote of full members 
at an election or a meeting called for this purpose.  It must also be 
approved by MathSoc.

\section{Relationships}

\subsection{Mathematics Society of the University of Waterloo}
The club is a part of the Mathematics Society, which is part of the Federation 
of Students. The club will receive financial support, recognition, and legal 
support through the Federation of Students if necessary. The club shall be the club 
that carries out activities of interest to Actuarial Science Students. 
The club shall attempt to maintain contact with the Actuarial Science Representatives 
of the Mathematics Society.

\subsection{University of Waterloo ASNA Delegates}
The club will provide publicity for ASNA through the Facebook group, and will 
offer the office for meetings.

\section{Keys}
Keys will be given to the President and Vice-Presidents.

\end{document}
