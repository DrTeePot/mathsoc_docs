\section{Mathematics Instructor of the Year Committee}
effective May 26, 2004; replaces December 4, 2002

\subsection{Composition}
The membership of the Mathematics Instructor of the Year Committee shall consist of the following:
\begin{enumerate}
\item The Vice-President, Academic, as Chair; and
\item One to three Council appointees.
\end{enumerate}


\subsection{Procedures}
The Mathematics Instructor of the Year Committee shall be formed within the first six weeks of Winter term. The Committee shall meet at least once after the close of nominations, further meetings being at the discretion of the Committee. All meetings of the Committee are to be closed and confidential.
At the beginning of each term, an announcement will be made regarding the existence of this award.
There shall be a period of no fewer than two weeks during which nominations can be received. This period shall not extend past the tenth week of classes. There shall be a notice of nominations posted in the Society office at that time. Nominations should include a description of the candidate's teaching technique.
If, at the first meeting after the close of nominations, the Committee feels that one or more deserving persons have been overlooked in that a nomination was not received for said persons, the Committee will:
\begin{enumerate}
\item Obtain a description of said person’s teaching ability; and
\item Consider said persons as nominees in company with the gathered nominations.
\end{enumerate}
The Committee will review nominations obtained in the past year and select one candidate to recommend to Council to receive the IOY award. The Committee may recommend any number of instructors to receive honourable mentions.
Nominations of Individuals not recommended by the committee for the award shall be maintained in confidence.
Notwithstanding the above, any information received about a nominee may, with agreement of the nominator, be forwarded to the Distinguished Teacher Award Committee.

\subsection{Criteria For Nominees}
The Nominee must:
\begin{enumerate}
\item Be a current instructor in the Mathematics Faculty.
\end{enumerate}

Note: As outlined in the Guidelines for the Selection Committee for the University of Waterloo Distinguished Teacher Awards: The Distinguished Teacher Award has been set up by the Senate of the University of Waterloo in recognition of the great importance of excellence in teaching at all levels in the University. The award is open to all those who teach students in the University of Waterloo and its federated and affiliated colleges. Recipients are to be chosen from among nominees by a Selection Committee of faculty and students."

The Instructor of the Year Award has been set up by the Mathematics Society of the University of Waterloo in recognition of the importance of presentation when instructing within the Mathematics Faculty. The award is open to only those instructors of full courses within the Mathematics Faculty. Recipients are to be chosen from the recommendations of the Instructor of the year committee based on nominations by students within the Mathematics Faculty.

Honourable mentions are not meant to decrease the prestige of the Instructor of the Year Award.  They are intended to make the Award more respected and useful.  Honourable mentions should not be taken lightly and should be kept to as few as possible.

There are times when the IOY Committee, when allowed, would be tempted to give a tie or wind up making their decision based on small enough factors that the runners up are still worthy of recognition.  It is very possible that in other years these runners up could have won.  These are the situations that honourable mentions are meant for.