\section{Clubs Policy}
\subsection{Overview}
This policy governs all clubs under the aegis of MathSoc. MathSoc establishes
clubs to promote social and academic relations inside programs, departments,
and/or other groups of interest inside the Faculty. All clubs are expected to
abide by legitimate decisions of MathSoc and of Council, and MathSoc provides
resources to the clubs in return.

The resources and status provided to a club are not a right, but a privilege.
As such, it is in MathSoc's discretion to continue or discontinue support and/or
recognition.

\subsubsection{Establishment of New Clubs}
Members seeking to establish a new club under MathSoc shall submit the following
to the Clubs Director, who shall present the petition to Council:

\begin{itemize}
  \item A contact email for the club;
  \item A constitution for the proposed club;
  \item A current list of executive officers and members for the proposed club;
  \item In the case of club for an academic program, the signature of a
    professor or professors who advises for the program(s), sponsoring the
    establishment of a club for that program; and
  \item For any other club, the signature of the Vice-President, Internal of the
    Federation, authorizing MathSoc to establish that club.
\end{itemize}

Upon request of a group seeking recognition for a new academic club for a
program that does not currently have a club, the Clubs Director shall assist the
group in advertising and holding an initial meeting (or series of meetings) to
adopt a constitution and elect executives.

Consideration of a petition for a new club at Council requires notice, and shall
be a general order at the meeting when notice has been provided. Council is not
obliged to accept the petition for the new club and may in particular request
additional information from the sponsors.

The Executive may exempt a newly-created club from deadlines outlined in this
policy in the term of their creation, as is appropriate to allow the club to
establish itself.

\subsubsection{Constitution Requirements}
If a club is to have any affiliation with any organization outside of the
University, this must be stated clearly in the constitution of the Club.

Unless the club is affiliated (or seeking affiliation) with another organization
whose rules require a lower (including zero) fee, then a club must charge a
termly membership fee of at least \$2, and this fee must be outlined in its
constitution. If the club is affiliated or seeking affiliation with another
organization with a lower maximum fee, then that maximum must be charged. The
membership fee need not be charged for a new club's first term.

A club must make full membership open to all MathSoc social members and must
restrict it to the same, unless it is affiliated (or seeking affiliation) with
another organization, in which case it may also allow full membership to be
available to members of that organization. A club may have other forms of
membership open to the University community. A club must restrict the privileges
to vote and hold executive positions to full members. A club must not practice
discrimination in its acceptance of members or of executive.

A club must elect its executive officers for a given term no later than the
third week of that term.

In the event of any conflict between a decision of MathSoc, including a policy,
and the constitution of a club, the decision of MathSoc shall prevail.

\subsubsection{Discipline}
Clubs are expected to behave as upstanding members of the University community
and to contribute to the purposes of the Society. Failure to do so may
constitute grounds for sanction by MathSoc.

A club may be put on probation either by a decision of Council or through the
operation of this policy. If a club is on probation, then the Clubs Director
must proactively monitor the club for violations of policies. At the meeting
after budget meeting of the term after the club was put on probation, and of
every term thereafter as long as the club remains on probation, Council shall
evaluate the probation and consider whether to take any action, including to
lift the probation or to impose sanctions.

All motions to discipline a club or to put a club on probation require notice,
given to the club executive directly as well as to Council, except for at the
meeting mentioned above with respect to clubs that have been on probation since
the start of the term.

Additionally, as outlined elsewhere in this policy, a club may have its funding
withheld as a consequence of failing to meeting a deadline.

\subsubsection{Disbandment}
A club may be disbanded by Council. Such a decision requires notice, given to
the club executive directly as well as to Council, and a two-thirds vote. A club
will normally only be disbanded for continuing to violate policies or other
decisions of Council while on probation, but the decision to disband a club may
be made by Council in its sole discretion at any time. A decision to disband a
club is not effective until the end of the Council meeting after it is adopted,
and becomes final at that time.

In the event that a club is disbanded, unless its constitution states otherwise,
its assets shall be transferred to MathSoc. Such a provision in a club's
constitution is ineffective unless approved by Council (which may happen before
the club is disbanded).

\subsection{Operations}
\subsubsection{Events}
All club events shall have event forms filed with the Federation. In accordance
with the Societies Agreement, clubs must receive approval for their events.  The
Vice-President, Internal and the Clubs Director shall assist clubs in filing
event forms and securing approval for events.

All promotional material for events must include MathSoc's logo.

A social event is an event with negligible academic purpose.

\subsubsection{Meetings}
Clubs must hold meetings at least once a term. The President and the Clubs
Director shall be permitted to attend all club meetings.

\subsubsection{Records}
A copy of the club constitution, a membership list, and all financial records
must be provided to the Clubs Director upon request. If a club fails to do so
within one week of the request, the club's funding shall be withheld until it
provides the requested records.

A Club shall provide a copy of its membership list to MathSoc at the end of each
term for financial calculations.

Along with its budget each term, a club shall submit a contact email, a list of
its executive members, and a list of the events it intends to run in that term.

If a club amends its constitution, it must provide an updated copy of the
constitution to the Clubs Director. If it has previously provided a link to an
online copy, it must inform the Clubs Director that the constitution has been
updated.

\subsubsection{Internal Discipline}
If a club engages in any process of internal discipline, including removing an
executive from office or barring an individual from its office, it shall inform
Council of the situation.

If a club's constitution does not provide for disciplinary measures, then a club
may apply to Council to have one of its executive removed from office or to
expel a member of the club.

\subsubsection{Council}
Clubs are expected to have representation at each meeting of Council, and it is
a club's responsibility to ensure that it has a representative at each meeting
so as to be informed of changes and to debate any motions that may arise.
Council is free to consider any business relating to clubs regardless of the
clubs' presence or absence, and a club is not excused from the effects of a
decision because it failed to send a representative.

All club executives are entitled to attend meetings of Council for business that
relates directly to their club or to clubs in general. An executive of a club
who is not a member of Council may, unless the club's president indicates
otherwise, use or share the club president's speaking turns on any business. If
the club president has speaking rights for another reason, they may speak with
those rights in addition to sharing their rights as president.

\subsection{Finances}
\subsubsection{Budget}
Each term, no later than the end of the third week of term, a club shall submit
a budget for the term for MathSoc. The budget must outline the club's expected
spending for the term. The budget must be accompanied by the records specified
elsewhere in this policy.

A club's budget must outline general areas of income and spending. Each social
event must appear as a separate line item. Council may amend the budget as it
sees fit before approving it.

If a club's membership increases over the course of the term, it may request
additional funding for social events up to its new social event cap. If the
club's account balance decreases over the course of the term, this shall not
affect its social event cap.

If a club fails to submit its complete budget package on time, its package shall
not be considered at the budget meeting, even if submitted before the meeting,
and the club's funding shall be withheld until its budget is approved by
Council. If a club fails to submit its complete budget package for an entire
term, it shall be put on probation.

Clubs are permitted to spend \$80 (or more, if part of their budget request) per
term for recruitment and elections.

\subsubsection{Management of Funds}
A club's funds may be managed by the club or by MathSoc, at the club's choice.

For large purchases, the Clubs Director and the VPF shall assist the club in
arranging for the purchase to be paid directly rather than by reimbursement.
Small expenses shall normally be reimbursed.  All expenses for a club must be
signed off on by one of the club's executive officers. If a club's funding is
being withheld, no expenses or reimbursements shall be paid for that club.

A club's social events cap is $\max(M(5 + .3F), 250) + R$ where $M$ is the
number of club members who are also MathSoc social members, $F$ is the
club's membership fee, and $R$ is any external revenues being used to fund social events. All expenses of a club that are approved by Council are
eligible for reimbursement unless otherwise directed by Council. In a given
term, social event expenses are eligible for reimbursement only up to the
minimum of that club's social events cap and its academic event expenses.

% Crafted with <3 by a number of accounting students
Reimbursements must be requested in the term that the expense is budgeted. 
An arrangement to submit the request in a future term should be made with
the Vice President, Finance if:
\begin{itemize}
	\item an expense is payable on or after the first day of the final exam period;
	\item a reimbursement will be received after the first day of the final exam period; or
	\item an expense has been made but a reimbursement will be submitted in a 
	future term (due to time or resource constraints).
\end{itemize}

\subsubsection{By MathSoc} 
If a club opts to have its funds managed by MathSoc, then it shall be given its
own account under MathSoc. The club may carry a negative balance in its account.
The club shall have access to its financial records at any time, including its
current balance, expenditures, and income.

The club may carry a cash float as needed, provided that it is properly
accounted for. The club shall reconcile their cash float with the VPF at least
once per month, except during exams.

Expenditures and income from and to the club shall be applied directly to the
club's account.

At the end of each term, MathSoc shall transfer to the club an amount equal to
its expenses eligible for reimbursement, except that the club's balance shall
not be brought above \$0. If a club's closing balance is positive, this shall
be noted as a revenue item on the next terms budget.

After this procedure, if the club still has a negative balance, the VPF shall
report this to Council. A club that has carried a negative balance for multiple
terms in a row, or that has a balance of less than $-\$1000$ shall be placed on
probation and should expect less funding in the following term.

A club may spend its own funds as approved by the club's executive officers. If
the VPF believes that this will cause the club to have a significantly negative
balance at the end of the term, then they shall bring this to Council's
attention before approving the expenses.

\subsubsection{By the Club}
If a club opts to manage its own funds, then it shall maintain proper accounting
records of all its finances, including funds, income, and expenditures, to the
satisfaction of the Federation's Societies Accountant and the VPF. The club
shall provide whatever financial records are requested within one week of the
request, or else its funding shall be withheld.

Normally, the club shall be expected to reimburse its members for its expenses
where possible. If not feasible, then the Clubs Director and the VPF shall
arrange with the club and the Societies Accountant to reimburse the member
directly, or to pay for the purchase directly.

At the end of each term, MathSoc shall reimburse the club for its remaining
eligible expenses during the term, except that a club's balance shall not be
brought above \$1000.
